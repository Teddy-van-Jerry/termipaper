\documentclass{refart}
\usepackage{newtxtext}
\usepackage{upquote}
\usepackage{cite}
\usepackage{tikz}
\usepackage[hidelinks, pdfusetitle]{hyperref}

\newcommand\TermiPaper{\textsc{TermiPaper}}
\newcommand\TPversion{v0.1.0}

\title{Managing Academic Papers in Your Terminal With \TermiPaper}
\author{Wuqiong Zhao\texorpdfstring{\\
  University of California San Diego\\
  La Jolla, CA, USA}{}}
\date{\TPversion\\\today}

\pagestyle{myfootings}
\markboth{Managing Academic Papers in Your Terminal With \TermiPaper}%
         {Managing Academic Papers in Your Terminal With \TermiPaper}

\begin{document}

\maketitle

\begin{abstract}
  This is the abstract.
\end{abstract}

This manual is currently under construction.

\tableofcontents

\section{Introduction}\label{sec:introduction}
Managing academic papers is a common task for researchers and students.

Papers like \cite{zhao2023ompl}...


\section{Installation}\label{sec:installation}
\subsection{Install Binary via Cargo}
If Rust Cargo is installed,
you can install \TermiPaper\ by running the following command:
\begin{verbatim}
cargo install termipaper
\end{verbatim}

\subsection{Build from Source}
\TermiPaper\ is written in Rust.
To build from source, you need to have Rust installed.


\section{Highlights}\label{sec:highlights}
\subsection{Manage Papers}

\subsection{Citation Generation}
Users can export citation as BibTeX.

\subsection{Full Text Search}

\subsection{Auto Completion}

\subsection{Sharing}


\section{Usage}\label{sec:usage}
\subsection{Commands}
Several commands:
\begin{description}
  \item[\texttt{termipaper add}] Add a new paper to the database.
    Metadata can be added manually, fetched from the web or local files,
    or inferred from the PDF file.
  \item[\texttt{termipaper list}] List papers in the database.
  \item[\texttt{termipaper remove}] Remove a paper from the database.
\end{description}

Detailed explanations are listed below.

\subsection{Add a Paper}\label{subsec:termipaper-add}
\marginlabel{\texttt{termipaper add}}
To add a new paper to the database, use the command \texttt{add}.

% \seealso{}


\section{Conclusion}\label{sec:conclusion}
\input{sections/conclusion}

\phantomsection
\section*{Acknowledgments}
\addcontentsline{toc}{section}{Acknowledgments}

The author would like to thank \textit{GitHub Copilot} for assisting in writing this document.
The author would also like to thank \textit{Hubert Partl} and \textit{Axel Kielhorn} for creating
the \texttt{refart} \LaTeX\ class for this manual.

\phantomsection
\addcontentsline{toc}{section}{References}
\bibliographystyle{IEEEtran}
\bibliography{IEEEabrv, ref}

\end{document}
